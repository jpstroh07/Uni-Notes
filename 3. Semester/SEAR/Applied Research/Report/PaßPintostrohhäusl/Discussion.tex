\section{Discussion}

This chapter will discuss all the found results that were shown in the previous chapter. This chapter will also answer our research question and our hypothesis.

\subsection{Architectural Discussion}

As we have seen in Table 1, there were a lot of architectural changes with DDR5. DDR5 is way faster while consuming 10\% less power. This probably will not affect the power consumption of a PC as a whole dramatically, but it can save unnecessary power output on laptops, that are dependent on a battery that lasts a long time. As it was with DDR3 to DDR4, the laptop industry might consider changing to DDR5 where it is possible to save that power. In addition to the less power consumption, the PMIC, that was on the motherboard the generations before, is now soldered onto the module itself. This allows the modules to operate at 1.1 \gls{V} due to better power management, while also reducing potential noise \parencite{ddr5_overview_kingston}.
\\
\\
The chip density also got a huge boost. Now, one single chip on the module can contain up to 64 GB of data. This will mostly affect data centres, as they need high capacity RAM for their servers. Samsung even announced, that with DDR5, it is possible for them to create a module that can contain 1 \gls{TB} \parencite{samsung_1tb}. If these excessive amounts of capacity, that are possible with DDR5, will also be available for the normal PC user, cannot be said confidently.
\\
\\
What is surprising, is that the latencies and CAS latencies are much higher with DDR5 than it was with DDR4. CAS latencies determine how many clock cycles happen to put data from the CPU onto the RAM module, so it can be read again. However, the latency, which is given in \gls{ns}, determines how long the delay is between a data request from the CPU and when the data at the CPU arrives \parencite{DDR4_DDR5_CAS}. The CAS latency is much higher than it was with DDR4, but it does not explain everything related to latency. As we can see, the latency in ns has also increased with DDR5, but not as much as it could be expected considering the increase of the CAS latency. Because of this and the much higher data rate, the higher latencies are not affecting the performance by making the RAM less responsive \parencite{RAM_Latency}.
\\
\\
What could be an influence to the higher latencies, are the reliability changes, that were made. As we already disclosed, the CRC, which checks for invalid data that is being inputted to the RAM chip \parencite{crc_def}, has been introduced to the read cycle of the module. Now, DDR5 performs a CRC check with every cycle, regardless if it is a read or a write cycle. Also, On-Die ECC was introduced to every DDR5 RAM module. While the CRC checks, if the data that are inputted or outputted are correct, On-Die ECC can correct incorrect or damaged data \parencite{ddr5_overview_kingston}. Before, ECC was only reserved for special ECC RAM, that was exclusively used in servers. Now with DDR5, every module gets On-Die ECC correction. These changes improve the reliability of data, but it is unlikely to make a difference in daily usage. This is because even with DDR4, the only errors that result in blue screens that can happen on Windows related to RAM are about broken RAM modules or misconfigured power management for the RAM modules, and not incorrect data \parencite{bluescreens}. 
\\
\\
As the last changes, the Bank Groups of each module have increased to 8 groups, while keeping the Banks per Group the same. A bank in a RAM chip is a field, that contains the stored data, while a bank group is a package that contains these fields. Each bank group transfers data to the CPU one at a time. Banks need to recover by “refilling” with data, before they can transfer data again \parencite{bank_groups}. With the high data rate that comes with DDR5, this can be a problem, because how can a bank group transfer data to the CPU, when the banks are not recovered yet? To accommodate this, the number of bank groups have been increased to 8. In addition to that, the channel that is used to transfer data to the CPU has been split up from a single 64 bit channel to a dual 32 bit channel. This is again improving latency and efficiency, because we have now two independently working channels, that have lower bits needed to transfer to the CPU \parencite{ddr5_overview_kingston}.
\\
\\
All these changes can improve system stability, which is mostly important in server applications. For a home PC user, the higher data rate and the lower power consumption are probably the most important changes. The other changes will not have a noticeable effect on the system of a home PC user, other than better performance.

\subsection{Performance Discussion}

The architectural changes contributed to a small performance boost across the CPU. 
\\
As we can see in Figures 1 and 2, when using the DDR5 kits, the system performs on average 5\% better both the multiple CPU test and the single CPU test. The only exception are the results in the DOOM Eternal test in Figure 2. Here, DDR4 performs better than DDR5. Again, we cannot check if this is true or an error in the test, because we cannot test these results by ourselves. For the other test results, it can be a decent upgrade to use DDR5 RAM instead of DDR4. When we take a look at Figure 1, we see that the i5 12600K can compete with the much stronger i7 12700K when using a DDR5 RAM kit. The i5 12600K even beat the i7 12700K on average FPS in Cyberpunk 2077. 
\\
Generally speaking, a 5\% performance boost in games can be expected when using DDR5 instead of DDR4.
\\
\\
When we now look at the performance graphs of the synthetic benchmarks, we also see, that DDR5 performs better. However, here are the differences not that clear like in the game benchmarks. The 3DMark test in Figure 3 shows that the CPUs performance improve as a whole when using especially fast RAM, because there is an around 300 Points jump between the 6400 MT/s RAM and the 4800 MT/s RAM. When we take a look at the PugetBench results in Figure 5, we see that the CPU performs around 2\% better. The 7-Zip test in Figure 4 is more interesting. The decompressing rate has increased by 2\%, while when we look at the compressing rate, we see that the CPU performs around 48\% better when using DDR5. That means that the CPU performs a bit better when using work applications like Photoshop, encoding videos for streaming or other creative work, while performing much better in handling files. 
\\
\\
This is also the reason we need to use tests for both gaming and work performance. Just looking at the graphs, of course it can be seen that DDR5 performs better than DDR4, but just from the gaming benchmarks, we cannot conclude how much better DDR5 performs in work applications. Gaming benchmarks only test for the gaming capabilities, without stressing the system hard enough to get a better insight of how well the system will perform in creative work or other applications. The same can be said the other way around. Just because we see a 48\% improvement in the compressing file benchmark, does not necessarily mean that our gaming performance also will increase that much. So it is always necessary to test for both gaming and work performance.

\subsection{Price-performance Discussion}

While there is an improvement in performance, we also need to check if the price-performance ratio is acceptable to answer our research question. As we can see in Table 2, the price-performance ratio is not speaking for DDR5. In the benchmarks, DDR4 performs with a better price-performance ratio, giving more performance per Euro spend. In Figure 6, we have also compared two RAM kits from the manufacturer Crucial, and we discovered that the DDR5 kit is around 16\% higher priced than the DDR4 kit. However, this does not compare to the average costs of the kits that were used for the benchmarks. As Table 2 shows, the average DDR4 kit in the benchmarks costs €107.75, while the average DDR5 kit costs €159.11 \parencite{used_kits}. 
\\
\\
We also do need to remember, that with DDR5, a new motherboard needs to be purchased, because DDR4 and DDR5 does not have the same socket. This will also increase the price that needs to be spent for DDR5. This also speaks, from a cost perspective, more for DDR4, if DDR4 is already in use.

\subsection{Conclusion}

To conclude, DDR5 is a small improvement in gaming performance, and depending on the use case, a bigger performance in work applications. 
We can answer our research question with “DDR5 provides a small performance in gaming and work applications over DDR5”. However, we cannot confirm our hypothesis, because we hypothetized that DDR5 also provides a better price-performance ratio, which it did not. If now someone is in a situation, where they want to upgrade their PC system, we probably would not recommend upgrading to DDR5 RAM, because from our perspective, it is not worth it to spend over €130 plus a new, most likely expensive motherboard for a small improvement. However, as DDR4 replaced DDR3, DDR5 will be the new RAM standard that more and more new CPUs will only be compatible with, as it is already with the new AMD Ryzen 7000 series is \parencite{Ryzen_7000_RAM_specs}. It also could be, because the technology is still pretty new, that it will be improved over the years to be more attractive than DDR4, but from today's perspective, we would not recommend upgrading to it, if that was our research topic.

\subsection{Further Research}

As we already have disclosed multiple times already in this report, the main problem was that we could not do the tests ourselves. This is because we did not have the necessary hardware, more specifically the latest CPUs of Intel and DDR5 RAM. To make this report more reliable and to improve the results of this report, the necessary hardware would be needed. 