\section{Introduction}

This chapter is going to provide the context and background of this report. Furthermore, the research question and the corresponding hypothesis will be defined.

\subsection{Context and Background}

This report is about comparing two generations of \gls{DDR} RAM, but before comparing these generations, we need to first talk about what exactly RAM means and what it does.
\\
A PC is a complex system, featuring a lot of different components. Among the key components, most importantly the \gls{CPU}, there is a specific amount of \gls{RAM} needed to function. The CPU is the centre of the PC, processing and calculating everything the PC needs to do. In order to work, the CPU needs to have access to RAM. The RAM, that has a direct connection to the CPU, functions as a quick access storage for running programs. It caches every running program, so that the CPU can access these running programs at any time \parencite{RAM_Definition}. A RAM module consists of two major parts: the RAM chips and the RAM module itself. The chips are soldered onto the RAM module and contain the stored data, while the module is the part that connects to the motherboard. RAM has many factors that determine how well the RAM module that is used will perform. The most important four of the factors are: the size of the RAM module, the clock speed, the latency, and the frequency. The size is measured in \gls{GB}s, and it determines how many processes can be stored simultaneously. The clock speed of a RAM module defines how many write-and-read-cycles the RAM module can perform in a second and is measured in MT/s \parencite{RAM_Speed}, while the latency defines how much delay in nanoseconds between the CPU and the RAM module are \parencite{RAM_Latency}. Also, RAM has a frequency in MHz, which defines how many clock cycles there are in a second \parencite{RAM_Speed}.
\\
\\
Now that we know what RAM is and how it functions, we will take a short glance at the history of RAM. Because the PC Industry is constantly improving, there are multiple generations of RAM available on the market.
\\
As a first attempt for system memory, \gls{SDRAM} was developed and used to synchronize with the timings of the CPU. However, SDRAM could only read once per clock cycle. As a successor, DDR RAM was introduced at the beginning of 2000. As the name says, DDR RAM achieves double the data rate of the SDRAM since it could read twice per clock signal. Thus, DDR RAM became much faster than SDRAM \parencite{RAM_generations}.
\\
The first generation was quickly superseded by the second generation of DDR RAM, called DDR2, in 2003. DDR2 had double the speed of the first generation thanks to an improved bus signal, which is a composite signal that consists out of other signals \parencite{DDR2_bus}. Thereby, DDR2 RAM had a significant higher bandwidth but still the same clock rate. Besides that, the new generation has a higher data rate, 533 - 800 MT/s compared to the 266 - 400 MT/s the generation had before. \gls{MT/s} refers to the number of data transferring operations in a second \parencite{RAM_Speed}. The lower voltage of DDR2 made the new generation even more efficient \parencite{RAM_generations}. 
\\
After around four years, in 2007, DDR3 RAM was introduced to the market. DDR3 RAM was more efficient and again faster than the previous generation of RAM. The needed voltage was around 40\% lower and offered a much more efficient way of operating. The lower voltages made it possible that the newer generation was usable for battery-powered devices, which meant also a huge improvement on the laptop market. In addition to that, the data rate has a speed of 1066 MT/s up to 1600 MT/s \parencite{DDR2_vs_DDR3}. 
\\
It took some time, but after seven years, in 2014, a new generation of RAM has been released. DDR4 RAM provides again with lower voltages and significantly higher transfer rates. DDR4 RAM can operate between 1600 MT/s and up to 3200 MT/s. In addition to that, a single DDR4-RAM module can have a capacity of up to 32 GB, allowing data centres equipping their servers with up to 1 TB of RAM \parencite{RAM_generations, RAM_generations_2}.
\\
\\
Last year, specifically at the end of 2021, the fifth and latest generation of DDR RAM was released to the marked. DDR5 again was on paper a performance boost for the PC market, doubling the clock rate to a maximum of 8800 MT/s while also providing a higher efficiency than DDR4 \parencite{ddr5_overview_kingston}. When considering these changes, it could be assumed that DDR5 is an improvement to DDR4, like DDR4 was to DDR3. However, when we take a look at the CPU market, we see that Intel produces their latest two CPU line-ups, the 12th and 13th generation of the Intel Core line-up, compatible with both DDR4 and DDR5 RAM \parencite{Intel_13_presentation, Intel_12_RAM_specs}, while AMD produces their latest CPUs, the AMD Ryzen 7000 series, only compatible with DDR5 RAM, while the older AMD Ryzen 5000 series only supports DDR4 RAM \parencite{Ryzen_5000_RAM_specs, Ryzen_7000_RAM_specs}. That made us wonder why Intel manufactures CPUs compatible with two RAM generations, while AMD does not bother. With the increased prices of DDR5 and the relative cheap costs of DDR4 \parencite{Geizhals_ddr4, Geizhals_ddr5}, this led us to question if DDR5 is really worth it and that much of an improvement to DDR4.

\subsection{Research Question and Hypothesis}

This report is going to determine if DDR5 is an improvement to DDR4. Because there are multiple ways to use a PC, there are different scenarios of how DDR5 could improve performance. The main scenarios are the work and gaming aspect. That led us to the research question: “How does DDR4 RAM and DDR5 RAM compare in gaming and work performance?”. Regarding the question and because there was every generation a boost in performance, we hypothesize that DDR5 is an improvement to DDR4 while providing a better price-performance ratio, in gaming as well as in work applications. We will also consider the architectural changes that were made to the modules itself, which cannot be captured by performance tests.