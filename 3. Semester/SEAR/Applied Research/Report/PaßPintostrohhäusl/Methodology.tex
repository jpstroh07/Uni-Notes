\section{Methodology}

This section is going to explain how we gathered and transformed our data, which will be used to answer our research question and our hypothesis.

\subsection{Gathering Data}

To have a better understanding of how we will show the results, we divided our comparative analysis into three major parts: One part will compare and analyse the architectural differences between DDR4 and DDR5, while the next part will show the results of performance tests. Also, there will be a section that will determine at what price-performance ratio DDR4 and DDR5 RAM operate.
\\
\\
Even though there are multiple RAM manufacturers, every RAM module inside their generation is built the same except for the amount of storage, the clock speed and the latency. For an architectural comparison, this does not matter, because these are the only values that have the most impact on the performance. The information we collected are: the range of data rates, the voltage that generation typically consumes, the range of the density of each RAM chip, the latency and \gls{CAS} latency, the bank groups and the banks per group, how the channel architecture between the CPU and RAM is configured, where the power to each RAM chip is distributed, at what cycles the \gls{CRC} is triggered and another architectural change that was made called On-Die \gls{ECC}. What all this means will be discussed in chapter 4, as this section is only to explain what data was gathered and how we transformed and visualized said data. For this comparison, we used the provided data sheets of reputable RAM chip manufacturers of Micron, SK Hynix and Samsung. The reputability of these manufacturers is measured by the market share in the \gls{DRAM} chip market, which these companies hold. The market share of these companies combined is at 94\%, while every company has a market share of at least 22\% on the global market \parencite{marketshare}.
\\
\\
The performance tests were more difficult than the comparison of the architecture. The big problem was that none of us owns the latest Intel Core CPUs or DDR5 RAM, so in that cases we need to rely on tests that were already done. Nonetheless, we can define what tests were done and what tools were used. Because we defined that we wanted to look into gaming and work performance, we first defined games that were used for the tests. The games are: Doom Eternal, F1 2020, Hitman 3, Assassin's Creed Valhalla, Shadow of the Tomb Raider, Red Dead Redemption 2, Cyberpunk 2077 and Forza Horizon 5. These games are good for this kind of comparisons, because these games are considered among the best games for a system benchmark \parencite{best_games}. These games are difficult for the GPU, but also for the CPU to process and in these cases, the CPU relies on good RAM. Every game has a built-in benchmark tool, that is used for the tests. These benchmarks will offer a variety of information, but what we are looking for are the average \gls{FPS} in said games. This value determines how well a game is playable by how smooth it is to play. When a PC performs well, it will get more FPS in a game. For the tests, different CPUs were used, because we wanted to see how much of a difference between DDR4 and DDR5 can be in different CPU generations. The CPUs that were used are the middle-tier Intel Core i5 12600K, the high-tier Intel Core i7 12700K and the best CPU of the 12th generation, the Intel Core i9 12900K. These CPUs support both DDR4 and DDR5 RAM, which is perfect to compare DDR4 RAM and DDR5 RAM.
\\
For the work aspect, benchmark tools like 3DMark, PugetBench and 7-Zip were used. 3DMark is a tool that stress tests the CPU as a whole to test how well the CPU performs in stress situations \parencite{3dmark_def}. PugetBench is designed for Adobe Photoshop, so it stress tests how well the CPU would perform in Adobe Photoshop, but the results can also be extended to general creative work \parencite{what_is_pugetbench}. 7-Zip can, besides compressing and decompressing ZIP-files, also be used as a benchmark to test how well the program can compress and decompress based on the performance of the CPU \parencite{7zip_def}. In these benchmarks, the only CPU that was used was the Intel Core i9 12900K, because for us, it was only interesting how the best CPU of the 12th generation performs when adapting the RAM. These benchmarks are, as a contrast to the gaming benchmark, synthetic, which means that they test the system at an unreasonable level of stress, which will likely never occur in a real world scenario. However, these benchmarks show how well a system can perform in general work applications, not only in a stress situation, which is exactly what we are looking for \parencite{benchmark_differences}. These benchmarks will give us numbers, which are calculated inside the benchmark tool itself and can be compared to other points of the same benchmark. The reason we tested for gaming performance and work performance and not just for gaming performance or just for work performance can be found in chapter 4. 
\\
To determine the price-performance ratio, we gathered the price of the RAM modules that were used in the gaming and work benchmarks. We got the prices from the manufacturers of the RAM modules and the prices that were listed on the sites of the vendors. 

\subsection{Data Transformation}

The data in the architectural comparison contains only the most important architectural features of DDR4 and DDR5. This data cannot be transformed in any way, because these are system specifications that are set. The only “transformation” we did was that we excluded how specifically data is stored and moved inside the DRAM chips, because that would go way beyond the scope of this report. Even though we specified before, that the data rate, the chip density and the latency do not matter in an architectural comparison, we felt like it was still worth mentioning at what ranges these type of specifications can vary, so we included these values in our comparison.
\\
The transformation for the performance comparison was a bit more difficult, because as we said before, we cannot test it, so we need to rely on third-party tests. We discovered while the difference between DDR4 and DDR5 was consistent, the FPS in the games were in some tests way higher or lower than in other games. To accommodate this problem, we calculated the average FPS of every test and saw, that the difference between DDR4 and DDR5 are still the same, so we can use this type of transformation. We are not strictly speaking dependent on the exact FPS in every game, as long as the difference is the same in every test.
\\
The benchmarks for the work aspect were transformed from their original form into the diagrams that will be displayed below. Here, we did not calculate any averages, because in these benchmarks, it is important to change the value as little as possible. This is because here we are dependent on the actual result, because unlike the gaming tests, these tests are not dependent on external influences, like what \gls{GPU} we use or what screen resolution we are using. These tests only stress the CPU, which is dependent on RAM.
\\
To get the data we used for our price-performance comparison, we first gathered the prices of each used RAM kit. Here we calculated the average cost of a DDR4 and DDR5 kit. After that, we calculated from every benchmark the average achieved points, so that we then can calculate how many points per Euro someone would get in each benchmark. We used these calculated values for our price-performance comparison.

\subsection{Data Visualization}

For the architectural and price-performance comparison, we created tables with every data we found or calculated to the specified categories above. This way we can easily see what differences in the architecture were made.
\\
To see how many FPS a specific configuration in a game achieved or how much points a configuration achieved in a benchmark, we created diagrams with the calculated data. The calculated data was put in Excel, so we created and designed the diagrams in Excel. The format of the diagrams in the performance comparison is called “Clustered Bars”, while the format for the price trend is called “Line”. We chose these types of formats, because with our data, it is better to understand our data while also being readable.
\\
We used the online \LaTeX -editor Overleaf for writing and styling this report.